\chapter{Conclusions}

The Standard Model of Particle Physics is robust at low energy physics, but is not entirely satisfactory.
A variety of observations, both theoretical and experimental, indicate that many of the problems identified with the Standard Model suggest new physics at the TeV scale.
Among these new physics models are those containing candidate dark matter particles, including supersymmetric extensions of the Standard Model.
Recently, elementary particle physicists have gained access to powerful new tools at CERN's Large Hadron Collider, including the CMS detector, that can use high energy proton collisions to probe the physics of the weak scale.
A pair of searches targeting pair-produced particles decaying to dark matter have been performed at CMS using its 137~\fbinv dataset of 13~TeV proton-proton collisions recorded from 2016 to 2018.
Both searches produce the most stringent constraints to date on a variety of models, especially supersymmetric models.
The constraints of the first search apply generally to any model in which pair-produced colored states decay semi-invisibly.
The second targets similar models that also include long-lived particles producing disappearing tracks, a challenging final state to study and one of increasing interest in recent years as null results in searches for supersymmetry have increased the minimum mass scale of supersymmetry.
This second model in particular projects to benefit significantly from increased luminosity, and will be able to extend the sensitivity of CMS to some supersymmetric models by as much as a TeV.
